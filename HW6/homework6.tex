\documentclass[submit]{harvardml}

\course{CS181-S22}
\assignment{Assignment \#6}
\duedate{11:59PM EST, April 29 2022}
\newcommand{\attr}[1]{\textsf{#1}}
\usepackage[OT1]{fontenc}
\usepackage{float}
\usepackage[colorlinks,citecolor=blue,urlcolor=blue]{hyperref}
\usepackage[pdftex]{graphicx}
\usepackage{subfig}
\usepackage{fullpage}
\usepackage{amsmath}
\usepackage{amssymb}
\usepackage{color}
\usepackage{todonotes}
\usepackage{listings}
\usepackage{common}
\usepackage{bm}
\usepackage{enumitem}
\usepackage{tikz}
\usepackage{xifthen}
\usepackage{soul}
\usepackage{framed}

\usepackage[mmddyyyy,hhmmss]{datetime}

\definecolor{verbgray}{gray}{0.9}

\lstnewenvironment{csv}{
  \lstset{backgroundcolor=\color{verbgray},
  frame=single,
  framerule=0pt,
  basicstyle=\ttfamily,
  columns=fullflexible}}{}

\newcommand{\mueps}{\mu_{\epsilon}}
\newcommand{\sigeps}{\sigma_{\epsilon}}
\newcommand{\mugam}{\mu_{\gamma}}
\newcommand{\siggam}{\sigma_{\gamma}}
\newcommand{\muzp}{\mu_{p}}
\newcommand{\sigzp}{\sigma_{p}}
\newcommand{\gauss}[3]{\frac{1}{2\pi#3}e^{-\frac{(#1-#2)^2}{2#3}}}


\begin{document}
\begin{center}
{\Large Homework 6: Inference in Graphical Models, MDPs}\\
\end{center}

\subsection*{Introduction}

In this assignment, you will practice inference in graphical models as
well as MDPs/RL.  For readings, we recommend \href{http://incompleteideas.net/book/the-book-2nd.html}{Sutton and Barto 2018, Reinforcement Learning: An Introduction}, \href{https://harvard-ml-courses.github.io/cs181-web-2017/}{CS181 2017 Lecture Notes}, and Section 10 and 11 Notes.

Please type your solutions after the corresponding problems using this
\LaTeX\ template, and start each problem on a new page.

Please submit the \textbf{writeup PDF to the Gradescope assignment `HW6'}. Remember to assign pages for each question.

Please submit your \textbf{\LaTeX\ file and code files to the Gradescope assignment `HW6 - Supplemental'}. 

You can use a \textbf{maximum of 2 late days} on this assignment.  Late days will be counted based on the latest of your submissions. 
\\

\newpage

\begin{problem}[Explaining Away + Variable Elimination 15 pts]

  In this problem, you will carefully work out a basic example with
  the ``explaining away'' effect. There are many derivations of this
  problem available in textbooks. We emphasize that while you may
  refer to textbooks and other online resources for understanding how
  to do the computation, you should do the computation below from
  scratch, by hand.

  We have three binary variables: rain $R$, wet grass $G$, and
  sprinkler $S$.
We  assume the following factorization of the joint distribution:
$$
\Pr(R,S,G) = \Pr(R)\Pr(S)\Pr(G\, |\, R, S).
  $$
  
  The conditional probability tables look like the
  following:
  \begin{eqnarray*}
    \Pr(R = 1) &= 0.25 \\
    \Pr(S = 1) &= 0.5 \\
    \Pr(G = 1 | R = 0 , S = 0 ) &= 0 \\
    \Pr(G = 1 | R = 1 , S = 0 ) &= .75 \\
    \Pr(G = 1 | R = 0 , S = 1 ) &= .75 \\
    \Pr(G = 1 | R = 1 , S = 1 ) &= 1
  \end{eqnarray*}
  
 
  \begin{enumerate}
    \item Draw the graphical model corresponding to the
      factorization. Are $R$ and $S$ independent?  [Feel free to use
      facts you have learned about studying independence in graphical models.]
    \item You notice it is raining and check on the sprinkler without
      checking the grass.  What is the probability that it is on?
    \item You notice that the grass is wet and go to check on the
      sprinkler (without checking if it is raining).  What is the
      probability that it is on?
    \item You notice that it is raining and the grass is wet.  You go
      check on the sprinkler.  What is the probability that it is on?
    \item What is the ``explaining away'' effect that is shown above?
    \end{enumerate}
    
Consider if we introduce a new binary variable, cloudy $C$, to the the original three binary variables such that the factorization of the joint distribution is now: 

$$
\Pr(C, R,S,G) = \Pr(C)\Pr(R|C)\Pr(S|C)\Pr(G\, |\, R, S).
$$

\begin{enumerate}
    \setcounter{enumi}{5}
    \item For the marginal distribution $\Pr(R)$, write down the variable elimination expression with the elimination ordering $S, G, C$ (where $S$ is eliminated first, then $G$, then $C$).
    \item For the marginal distribution $\Pr(R)$, write down the variable elimination expression with the elimination ordering $C,G,S$.
    \item Give the complexities for each ordering. Which elimination ordering takes less computation?
\end{enumerate}
\end{problem}

\textbf{Solution:}
\begin{enumerate}
    \item 
    If $G$ is not observed, $R - G - S$ is blocked, $R$ and $S$ are independent by d-separation rules.
    \begin{center}
    \begin{tikzpicture}[
      node distance=1 cm and .5cm,
      bn/.style={draw,ellipse,text width=2cm,align=center}
        ]
        \node[bn] (eef) {\attr{G}};
        \node[bn, above right=of eef] (t) {\attr{S}};
        \node[bn, above left=of eef] (f) {\attr{R}};
        \path (t) edge[-latex] (eef)
        (f) edge[-latex] (eef);
        \end{tikzpicture}
    \end{center}
    
    \item 
    \begin{align}
        P(S = 1 | R = 1) &= \frac{P(S=1, R=1)}{P(R=1)} \\
        &= \frac{P(G=0 | S=1, R=1)P(R=1)P(S=1) + P(G=1 | S=1, R=1)P(R=1)P(S=1)}{P(R=1)} \\
        &= \frac{(0)(0.25)(0.5) + (1)(0.25)(0.5)}{(0.25)} = 0.5
    \end{align} 
    
    \item 
    \begin{align}
        P(S=1 | G=1) &= \frac{P(S=1, G=1)}{P(G=1)} \\
        &= \frac{P(G=1 | S=1, R=0)P(S=1)P(R=0) + P(G=1 | S=1, R=1)P(S=1)P(R=1)}{0.5} \\
        &= \frac{(0.75)(0.5)(0.75) + (1)(0.5)(0.25)}{0.5} = 0.8125
    \end{align}
    
    
    \item
    \begin{align}
        P(S=1 | R=1, G=1) &= \frac{P(S=1, R=1, G=1)}{P(R=1, G=1)} \\
        &= \frac{P(R=1)P(S=1)P(G=1|R=1, S=1)}{P(G=1 | S=0, R=1)P(R=1)P(S=0) + P(G=1 | S=1, R=1)P(R=1)P(S=1)} \\
        &= \frac{(0.25)(0.5)(1)}{(0.75)(0.5)(0.25) + (1)(0.5)(0.25)} = 0.5714 
    \end{align}
    
    \item Given that the grass is wet and knowing that it is raining makes the probability of the sprinkler being on lower compared to only knowing that the grass is wet and having no information on if it is raining, because we are ``explaining away" why grass is wet (via rain).
    
    \item
    \begin{align}
        P(R) &= \sum_C \sum_G \sum_S P(C) P(R | C) P(S | C) P(G | R,S) \\
        &= \sum_C P(C) P(R | C) \sum_G \sum_S P(G | R, S) P(S | C)
    \end{align}
    
    \item
    \begin{align}
        P(R) &= \sum_S \sum_G \sum_C P(C) P(R | C) P(S | C) P(G | R,S) \\
        &= \sum_S \sum_G P(G | R,S) \sum_C P(C) P(R | C) P(S | C)
    \end{align}
    
    \item 6. takes $2^3 = 8$ steps and 7. takes $2^4 = 16$ steps. Thus, 7. elimination ordering takes less computation.

\end{enumerate}


\newpage
\begin{problem}[Policy and Value Iteration, 15 pts]

This question asks you to implement policy and value iteration in a
simple environment called Gridworld.  The ``states'' in Gridworld are
represented by locations in a two-dimensional space.  Here we show each state and its reward:

\begin{center}
\includegraphics[width=3in]{gridworld.png}
\end{center}
The set of actions is \{N, S, E, W\}, which corresponds to moving north (up), south (down), east (right), and west (left) on the grid.
Taking an action in Gridworld does not always succeed with probability
$1$; instead the agent has probability $0.1$ of ``slipping'' into a
state on either side, but not backwards.  For example, if the agent tries to move right from START, it succeeds with probability 0.8, but the agent may end up moving up or down with probability 0.1 each. Also, the agent cannot move off the edge of the grid, so moving left from START will keep the agent in the same state with probability 0.8, but also may slip up or down with probability 0.1 each. Lastly, the agent has no chance of slipping off the grid - so moving up from START results in a 0.9 chance of success with a 0.1 chance of moving right.

Also, the agent does not receive the reward of a state immediately upon entry, but instead only after it takes an action at that state. For example, if the agent moves right four times (deterministically, with no chance of slipping) the rewards would be +0, +0, -50, +0, and the agent would reside in the +50 state. Regardless of what action the agent takes here, the next reward would be +50.

Your job is to implement the following three methods in file \texttt{T6\_P2.ipynb}. Please use the provided helper functions \texttt{get\_reward} and \texttt{get\_transition\_prob} to implement your solution.

\emph{Do not use any outside code.  (You may still collaborate with others according to the standard collaboration policy in the syllabus.)}  

\emph{Embed all plots in your writeup.}
\end{problem}
\newpage

\begin{framed}
\textbf{Problem 2} (cont.)\\

\textbf{Important: } The state space is represented using integers, which range from 0 (the top left) to 19 (the bottom right). Therefore both the policy \texttt{pi} and the value function \texttt{V} are 1-dimensional arrays of length \texttt{num\_states = 20}. Your policy and value iteration methods should only implement one update step of the iteration - they will be repeatedly called by the provided \texttt{learn\_strategy} method to learn and display the optimal policy. You can change the number of iterations that your code is run and displayed by changing the $\texttt{max\_iter}$ and $\texttt{print\_every}$ parameters of the $\texttt{learn\_strategy}$ function calls at the end of the code.

Note that we are doing infinite-horizon planning to maximize the expected reward of the traveling agent. For parts 1-3, set discount factor $\gamma = 0.7$.

\begin{itemize}
    \item[1a.]  Implement function \texttt{policy\_evaluation}.  Your
      solution should learn value function $V$, either using a closed-form expression or iteratively using
      convergence tolerance $\texttt{theta = 0.0001}$ (i.e., if
      $V^{(t)}$ represents $V$ on the $t$-th iteration of your policy
      evaluation procedure, then if $|V^{(t + 1)}[s] - V^{(t)}[s]|
      \leq \theta$ for all $s$, then terminate and return $V^{(t + 1)}$.)

    \item[1b.] Implement function \texttt{update\_policy\_iteration} to update the policy \texttt{pi} given a value function \texttt{V} using \textbf{one step} of policy iteration.
    
    \item[1c.] Set \texttt{max\_iter = 4}, \texttt{print\_every = 1} to show the learned value function and the associated policy for the first 4 policy iterations. Do not modify the plotting code. Please fit all 4 plots onto one page of your writeup.
    
    \item [1d.] Set \texttt{ct = 0.01} and increase \texttt{max\_iter} such that the algorithm converges. Include a plot of the final learned value function and policy. How many iterations does it take to converge? Now try \texttt{ct = 0.001} and \texttt{ct = 0.0001}. How does this affect the number of iterations until convergence?
      
    \item [2a.] Implement function
      \texttt{update\_value\_iteration}, which performs \textbf{one step} of value iteration to update \texttt{V}, \texttt{pi}.
      
    \item [2b.] Set \texttt{max\_iter = 4}, \texttt{print\_every = 1} to show the learned value function and the associated policy for the first 4 value iterations. Do not modify the plotting code. Please fit all 4 plots onto one page of your writeup.
    
    \item [2c.] Set \texttt{ct = 0.01} and increase \texttt{max\_iter} such that the algorithm converges. Include a plot of the final learned value function and policy. How many iterations does it take to converge? Now try \texttt{ct = 0.001} and \texttt{ct = 0.0001}. How does this affect the number of iterations until convergence?
    
    \item[3] Compare and contrast the number of iterations, time per iteration, and overall runtime between policy iteration and value iteration. What do you notice?
    
    \item[4] Plot the learned policy with each of $\gamma \in (0.6,0.7,0.8,0.9)$. Include all 4 plots in your writeup. Describe what you see and provide explanations for the differences in the observed policies. Also discuss the effect of gamma on the runtime for both policy and value iteration.
    
    \item[5] Now suppose that the game ends at any state with a positive reward, i.e. it immediately transitions you to a new state with zero reward that you cannot transition away from. What do you expect the optimal policy to look like, as a function of gamma? Numerical answers are not required, intuition is sufficient.
 
\end{itemize}
\end{framed}

\textbf{Solution:}
\begin{itemize}
    \item[1a.] Completed!
    \item[1b.] Completed!
    \item[1c.]
    \begin{figure}[h!]
        \includegraphics[width=0.5\textwidth]{HW6/Policy_1.png}
        \includegraphics[width=0.5\textwidth]{HW6/Policy_2.png}
        \includegraphics[width=0.5\textwidth]{HW6/Policy_3.png}
        \includegraphics[width=0.5\textwidth]{HW6/Policy_4.png}
        \caption{1c. Final learned value function and policy plots}
    \end{figure}
    \item[1d.]
     \begin{figure}[h!]
        \includegraphics[width=0.5\textwidth]{HW6/Policy_5.png}
        \caption{1d. Final learned value function and policy for $ct = 0.01, 0.001, 0.0001$}
    \end{figure}
    For $ct = 0.01$, it took 5 iterations to converge. \\
    For $ct = 0.001$, it took 5 iterations to converge. \\
    For $ct = 0.0001$, it took 5 iterations to converge. \\
   

    While broadly-speaking, making $ct$ smaller increases the number of iterations until convergence, in our case, once $ct$ is small, having $ct$ approach closer and closer to $0$ will not change our number of iterations until convergence. Our criteria for determining that the algorithm has converged is when $ct$ is greater than the largest difference between the current value and previous value amongst all possible states; on our fifth iteration, our values are not changed and the greatest difference between current and previous values amongst all states is $0$. Thus, for any small $ct > 0$, our number of iterations will be 5. 

    \item[2a.] Completed!
    \item[2b.] 
        \begin{figure}[h!]
        \includegraphics[width=0.5\textwidth]{HW6/Value_1.png}
        \includegraphics[width=0.5\textwidth]{HW6/Value_2.png}
        \includegraphics[width=0.5\textwidth]{HW6/Value_3.png}
        \includegraphics[width=0.5\textwidth]{HW6/Value_4.png}
        \caption{2b. Final learned value function and policy plots}
    \end{figure}
    \item[2c.]
    \begin{figure}[h!]
        \includegraphics[width=0.5\textwidth]{HW6/Value_25.png}
        \includegraphics[width=0.5\textwidth]{HW6/Value_31.png}
        \includegraphics[width=0.5\textwidth]{HW6/Value_38.png}
        \caption{2c. Final learned value function and policy plots for $ct = 0.01, 0.001, 0.0001$ respectively}
    \end{figure}
    For $ct = 0.01$, it took 25 iterations to converge. \\
    For $ct = 0.001$, it took 31 iterations to converge. \\
    For $ct = 0.0001$, it took 38 iterations to converge.\\
    We see that as $ct$ decreases, the number of iterations until convergence increases. This makes sense, since a smaller $ct$ demands that our previous and current value function be closer and closer in their maximal state distance, which calls for a greater number of updating values (ie greater number of iterations).
    
\\
\\
\\
\newpage
    \item[3.] For policy iteration, the number of iterations is lower until the algorithm converges, the time per iteration is smaller, and the overall runtime is faster, when compared to value iteration.
    \item[4.]
    Choosing $ct = 0.001$ and letting both algorithms run until convergence, we have the following plots for policy iteration and value iteration for $\gamma = 0.6, 0.7, 0.8, 0.9$.
    \begin{itemize}
        \item $\gamma = 0.6$
        \begin{itemize}
            \item Policy Iteration took 4 iterations
            \item Value Iteration took 27 iterations
        \end{itemize}
        \item $\gamma = 0.7$
        \begin{itemize}
            \item Policy Iteration took 5 iterations
            \item Value Iteration took 38 iterations
        \end{itemize}
        \item $\gamma = 0.8$
        \begin{itemize}
            \item Policy Iteration took 6 iterations
            \item Value Iteration took 59 iterations
        \end{itemize}
        \item $\gamma = 0.9$
        \begin{itemize}
            \item Policy Iteration took 6 iterations
            \item Value Iteration took 124 iterations
        \end{itemize}
    \end{itemize}
    \begin{figure}[h!]
        \includegraphics[width=0.4\textwidth]{HW6/Gamma_Policy_4.png}
        \includegraphics[width=0.4\textwidth]{HW6/Gamma_Policy_5.png}
        \includegraphics[width=0.4\textwidth]{HW6/Gamma_Policy_6.png}
        \includegraphics[width=0.4\textwidth]{HW6/Gamma_Policy_6.png}
        \caption{4. Policy Iteration plots for $\gamma = 0.6, 0.7, 0.8, 0.9$ respectively (last two plots are the same)}
    \end{figure}
    \begin{figure}[h!]
        \includegraphics[width=0.4\textwidth]{HW6/Gamma_Value_27.png}
        \includegraphics[width=0.4\textwidth]{HW6/Gamma_Value_38.png}
        \includegraphics[width=0.4\textwidth]{HW6/Gamma_Value_59.png}
        \includegraphics[width=0.4\textwidth]{HW6/Gamma_Value_124.png}
        \caption{4. Value Iteration plots for $\gamma = 0.6, 0.7, 0.8, 0.9$ respectively}
    \end{figure}
    We see that for both policy and value iteration, that as $\gamma$ increases, the number of iterations until convergence increases. This makes sense, since by having a larger discounting factor, we weight the expected value over all reachable next states more heavily, which means that every iteration our $V*$ changes more dynamically. Since there is greater variance between the current and past value function in each iteration due to a larger discounting factor, it will take more iterations for our algorithm to converge and meet the constraint that the greatest difference between current and previous values amongst all states is less than $ct$ will take more iterations to achieve. \\
    For both policy and value iteration, whereas increasing $\gamma$ leads to more iterations until convergence, which leads to a longer runtime, when making both algorithms run a constant number of iterations and then changing $gamma$, as $\gamma$ increases, the total runtime decreases.  
    \item[5.] Now that the game ends at any state with a positive reward, we expect our optimal policy at nonzero states to lead towards the closest positive reward state, while the three positive reward states (states 0, 4, and 14) will want to move towards a boundary to maximize the likelihood of staying in the same state next iteration and reaping the positive reward. As $\gamma$ increases, the policy will more strongly point towards leading states towards any of the three positive states as opposed to the dominant positive state 14, since the expected reward of reaching the less positive states is more strongly discounted.
\end{itemize}


\begin{problem}[Reinforcement Learning, 20 pts]
  In 2013, the mobile game \emph{Flappy Bird} took the world by storm. You'll be developing a Q-learning agent to play a similar game, \emph{Swingy Monkey} (See Figure~\ref{fig:swingy}).  In this game, you control a monkey that is trying to swing on vines and avoid tree trunks.  You can either make him jump to a new vine, or have him swing down on the vine he's currently holding.  You get points for successfully passing tree trunks without hitting them, falling off the bottom of the screen, or jumping off the top.  There are some sources of randomness: the monkey's jumps are sometimes higher than others, the gaps in the trees vary vertically, the gravity varies from game to game, and the distances between the trees are different.  You can play the game directly by pushing a key on the keyboard to make the monkey jump.  However, your objective is to build an agent that \emph{learns} to play on its own. 
  
   You will need to install the \verb|pygame| module
  (\url{http://www.pygame.org/wiki/GettingStarted}).
  

\textbf{Task:}
Your task is to use Q-learning to find a policy for the monkey that can navigate the trees.  The implementation of the game itself is in file \verb|SwingyMonkey.py|, along with a few files in the \verb|res/| directory.  A file called \verb|stub.py| is the starter code for setting up your learner that interacts with the game.  This is the only file you need to modify (but to speed up testing, you can comment out the animation rendering code in \verb|SwingyMonkey.py|). You can watch a YouTube video of the staff Q-Learner playing the game at \url{http://youtu.be/l4QjPr1uCac}.  It figures out a reasonable policy in a few dozen iterations.
You'll be responsible for implementing the Python function  \verb|action_callback|. The action callback will take in a dictionary that describes the current state of the game and return an action for the next time step.  This will be a binary action, where 0 means to swing downward and 1 means to jump up.  The dictionary you get for the state looks like this:
\begin{csv}
{ 'score': <current score>,
  'tree': { 'dist': <pixels to next tree trunk>,
            'top':  <height of top of tree trunk gap>,
            'bot':  <height of bottom of tree trunk gap> },
  'monkey': { 'vel': <current monkey y-axis speed>,
              'top': <height of top of monkey>,
              'bot': <height of bottom of monkey> }}
\end{csv}
All of the units here (except score) will be in screen pixels. Figure~\ref{fig:swingy-ann} shows these graphically. 
Note that since the state space is very large (effectively continuous), the monkey's relative position needs to be discretized into bins. The pre-defined function \verb|discretize_state| does this for you.

\textbf{Requirements}
\\
\textit{Code}: First, you should implement Q-learning with an
$\epsilon$-greedy policy yourself. You can increase the performance by
trying out different parameters for the learning rate $\alpha$,
discount rate $\gamma$, and exploration rate $\epsilon$. \emph{Do not use outside RL code for this assignment.} Second, you should use a method of your choice to further improve the performance. This could be inferring gravity at each epoch (the gravity varies from game to game), updating the reward function, trying decaying epsilon greedy functions, changing the features in the state space, and more. One of our staff solutions got scores over 800 before the 100th epoch, but you are only expected to reach scores over 50 before the 100th epoch. {\bf Make sure to turn in your code!} \\\\
\textit{Evaluation}: In 1-2 paragraphs, explain how your agent performed and what decisions you made and why. Make sure to provide evidence where necessary to explain your decisions. You must include in your write up at least one plot or table that details the performances of parameters tried (i.e. plots of score vs. epoch number for different parameters).
\\\\
\textit{Note}: Note that you can simply discretize the state and action spaces and run the Q-learning algorithm. There is no need to use complex models such as neural networks to solve this problem, but you may do so as a fun exercise.

\end{problem}
\begin{figure}[H]
    \centering%
    \subfloat[SwingyMonkey Screenshot]{%
        \includegraphics[width=0.48\textwidth]{figures/swingy}
        \label{fig:swingy}
    }\hfill
    \subfloat[SwingyMonkey State]{%
        \includegraphics[width=0.48\textwidth]{figures/swingy-ann}
        \label{fig:swingy-ann}
    }
    \caption{(a) Screenshot of the Swingy Monkey game.  (b) Interpretations of various pieces of the state dictionary.}
\end{figure}
    
\textbf{Solution:}
After implementing Q-Learning in stub.py, I was able to explore the parameter space in order to find the $\alpha, \gamma, \epsilon$ which has the best swingy monkey performance. Allowing the agent to learn over 100 epochs, metric of interest for me was the average score of the agent from epochs 50-100 (since earlier epochs are more likely to have low scores as the agent is still learning), as well as the highest score the agent achieved over all epochs. \\
After some initial exploration, I found that the agent performed best for small $\alpha$ (since $\alpha$ that was too big would change our learned $Q$ too heavily and lead too poor long-term performance). Meanwhile, I found that a large $\gamma$ performed best (since $\gamma$ that was too small would not put enough emphasis on the expected future value from the next state). Finally, I found that a small $\epsilon$ is the best (since given that there are only two actions, the agent is able to explore the state space fairly quickly over multiple iterations, and having too much random exploration will encourage the monkey to crash too often when the agent's current optimal action is correct.    \\
Once having a rough ballpark, I did a finer exploration of the learner parameters which had the best performance. The table below encompasses the results of these metrics based on changes in my values for $\alpha, \gamma, \epsilon$:

\begin{center}
\begin{tabular}{||c c c c c||} 
 \hline
 $\alpha$ & $\gamma$ & $\epsilon$ & Average score from epochs 50-100 & Highest score over all epochs \\ 
 \hline\hline
 0.1 & 0.6 & 0.1 & 0.98 & 13 \\ 
 \hline
 0.2 & 0.6 & 0.1 & 0.56 & 10 \\
 \hline
 0.3 & 0.6 & 0.1 & 0.58 & 7 \\
 \hline
 0.1 & 0.7 & 0.1 & 2.45 & 23 \\ 
 \hline
 0.2 & 0.7 & 0.1 & 0.75 & 10 \\
 \hline
 0.3 & 0.7 & 0.1 & 0.75 & 10 \\ 
 \hline
 0.1 & 0.8 & 0.1 & 1.45 & 10 \\ 
 \hline
 0.2 & 0.8 & 0.1 & 1.15 & 11 \\
 \hline
 0.3 & 0.8 & 0.1 & 0.82 & 9 \\ 
 \hline
 \hline
 0.1 & 0.6 & 0.01 & 2.58 & 43 \\ 
 \hline
 0.2 & 0.6 & 0.01 & 5.51 & 128 \\
 \hline
 0.3 & 0.6 & 0.01 & 12.67 & 238 \\
 \hline
 0.1 & 0.7 & 0.01 & 3.31 & 74 \\ 
 \hline
 0.2 & 0.7 & 0.01 & 12.47 & 159 \\
 \hline
 0.3 & 0.7 & 0.01 & 1.39 & 104 \\ 
 \hline
 0.1 & 0.8 & 0.01 & 15.47 & 195 \\ 
 \hline
 0.2 & 0.8 & 0.01 & 16.15 & 167 \\
 \hline
 0.3 & 0.8 & 0.01 & 1.62 & 52 \\ 
 \hline
 \hline
 0.1 & 0.6 & 0.001 & 25.33 & 494 \\ 
 \hline
 \textbf{0.2} & \textbf{0.6} & \textbf{0.001} & \textbf{69.83} & \textbf{1088} \\
 \hline
 0.3 & 0.6 & 0.001 & 40.22 & 645 \\
 \hline
 0.1 & 0.7 & 0.001 & 63.15 & 828 \\ 
 \hline
 0.2 & 0.7 & 0.001 & 50.33 & 363 \\
 \hline
 0.3 & 0.7 & 0.001 & 19.67 & 247 \\ 
 \hline
 0.1 & 0.8 & 0.001 & 15.52 & 914 \\ 
 \hline
 0.2 & 0.8 & 0.001 & 47.78 & 680 \\
 \hline
 {0.3} & {0.8} & {0.001} & {40.01} & {1048} \\ 
 \hline
\end{tabular}
\end{center}
From our results, we see that the optimal parameters for performance are $\alpha = 0.2, \gamma = 0.6, \epsilon = 0.001$ with respect to highest average score from epochs 50-100, which in this case was an average score of 59.23, as well as highest overall score of 1088. \\
We see that making $\epsilon$ smaller had the biggest impact in improving our agents performance, as the exploration was minimized and our agent was able to act optimally more often. Although $\epsilon$-greedy exploration is important in order to explore new states, since we only have two actions, it is easy to explore new states in this case and more beneficial to stop adding random exploration which sabotages the agent's optimal action which is likely correct. \\

For the second part to optimize our Q-Learning implementation, I have chosen set the learning rate $\alpha(s,a) = \frac{1}{N(s, a)}$, where $N(s, a)$ is the number of times action a is taken in state s. This ensures that we are decaying our learning rate over time. As long as we Visit every action in every state infinitely often, since we now decay the learning rate over time, Q-learning is guaranteed to converge to the optimal Q-values. \\
Using this time-varying definition for $\alpha$, and choosing $\epsilon = 0.001$ since it demonstrated a much better performance of the agent regardless of the values of $\gamma$ and $\alpha$ in the first part, we now have the following agent performance metrics for the same range of $\gamma$:
\begin{center}
\begin{tabular}{||c c c c||} 
 \hline
 $\gamma$ & $\epsilon$ & Average score from epochs 50-100 & Highest score over all epochs \\ 
 \hline\hline
 0.6 & 0.001 & 84.80 & 922 \\
 \hline
 0.7 & 0.001 & 77.12 & 496 \\ 
 \hline
 0.8 & 0.001 & 86.21 & 1005 \\ 
 \hline
\end{tabular}
\end{center}
We see that the average score from epochs 50-100 is much higher with temporal learning rate decay as opposed to without! We also see that the highest score over all epochs is just as large as our first version of Q-Learning (although this metric is less meaningful compared to the average score from epochs 50-100).

\newpage
\newpage
%%%%%%%%%%%%%%%%%%%%%%%%%%%%%%%%%%%%%%%%%%%%%
% Name and Calibration
%%%%%%%%%%%%%%%%%%%%%%%%%%%%%%%%%%%%%%%%%%%%%
\newpage
\subsection*{Name}
Arnav Srivastava
\subsection*{Collaborators and Resources}
Whom did you work with, and did you use any resources beyond cs181-textbook and your notes?
No and No.
\subsection*{Calibration}
Approximately how long did this homework take you to complete (in hours)? 
It took me many hours to complete this homework. But Swingy Monkey is my rock. Thank you for this.
\end{document}